\chapter*{Cuestionario} % Capítulo sin número
\addcontentsline{toc}{chapter}{Cuestionario Bloque 1} % Añadirlo manualmente al índice
\setchapterimage{orange9.jpg} % Chapter heading image
\chapterspaceabove{6.75cm} % Whitespace from the top of the page to the chapter title on chapter pages
\chapterspacebelow{7.25cm} % Amount of vertical whitespace from the top margin to the start of the text on chapter pages
\pagestyle{empty}
\begin{nosangria}
\Large{\textbf{Elige UNA respuesta para cada una de las preguntas} [18 min]}
\end{nosangria}

\begin{enumerate}
\item ¿Qué información NO se obtiene gracias a los datos del aire?

	\begin{itemize}
		\item[A)] La velocidad del avión.
		\item [B)] La altitud del avión.
		\item [C)] La dirección que apunta el morro del avión.\\
	\end{itemize}

\item ¿Si la temperatura del aire exterior es mayor a la estimada por la ISA?

	\begin{itemize}
		\item[A)] La densidad real es la misma que la estimada.
		\item [B)] La densidad real es menor a la estimada.
		\item [C)] La densidad real es mayor a la estimada.\\
	\end{itemize}


\item ¿Cuales son los componentes e instrumentos principales que incluyen un sistema pitot - estatica del avión? [1 min]

    \begin{itemize}
    \item[A)] El sistema pitot-estática incluye el velocímetro, altímetro, indicador de velocidad vertical, la brújula y el horizonte artificial.
    \item[B)] El sistema pitot-estática incluye el velocímetro, altímetro, indicador de velocidad vertical, puerto estático y tubo pitot.
    \item[C)] El sistema pitot-estática incluye el indicador de actitud, velocímetro, altímetro, indicador de viraje y puerto de presión dinámica.\\
     \end{itemize}

\item ¿Cual será el efecto sobre la densidad de una masa de aire si la temperatura desciende pero la presión permanece constante? [1 min]

    \begin{itemize}
    \item[A)] La densidad del aire aumentará al disminuir la temperatura, ya que las moléculas se comprimen.
     \item[B)] La densidad del aire disminuirá al descender la temperatura, ya que el volumen del aire aumenta.
    \item[C)] La densidad del aire se mantendrá constante, dado que la presión no cambia.\\
     \end{itemize}

\item ¿Que se entiende por ''atmosfera estandar''? [1 min]
    \begin{itemize}
    \item[A)] La atmósfera estándar es una referencia utilizada en aeronáutica que considera condiciones ideales de temperatura, presión y densidad a diferentes altitudes.
    \item[B)] La atmósfera estándar es el estado de la atmósfera en condiciones de alta presión y baja temperatura, típicamente en la superficie del mar.
    \item[C)] La atmósfera estándar se refiere a la cantidad de vapor de agua en el aire, que se considera constante en todas las altitudes.\\
     \end{itemize}

\item ¿Cuál es la diferencia entre la ''altitud de presión'' y ''altitud indicada''? [1 min]
\begin{itemize}
    \item[A)] La altitud de presión se basa en la presión atmosférica medida en el altímetro, mientras que la altitud indicada se refiere a la altitud calculada en relación con el nivel medio del mar.
    \item[B)] La altitud de presión se refiere a la altitud real del avión en el aire, mientras que la altitud indicada es la altitud ajustada por los cambios en la temperatura.
    \item[C)] La altitud de presión es la altitud corregida por la temperatura, mientras que la altitud indicada es la lectura del altímetro sin correcciones.\\
\end{itemize}

\item ¿Qué efecto tiene una reducción en la presión atmosférica en la altitud indicada en un avión? [1 min]
    \begin{itemize}
    \item[A)] La altitud indicada aumentará.
    \item[B)] La altitud indicada disminuirá.
    \item[C)] No habrá efecto en la altitud indicada.\\
    \end{itemize}

\item ¿Cuál de las siguientes afirmaciones es correcta? [1 min]
    \begin{itemize}
    \item[A)] El anemómetro trabaja exclusivamente con presión dinámica.
    \item[B)] El altímetro trabaja exclusivamente con presión estática.
    \item[C) ]El funcionamiento del VSI se basa en la medición de la variación de la presión de remanso.\\
    \end{itemize}  
    
 \item ¿Qué modelo se utiliza para la calibración de los instrumentos de presión? [1 min]
\begin{itemize}
    \item[A)] Atmósfera real.
    \item[B)] Atmosfera Estándar internacional (ISA).
    \item[C)] Atmosfera dinámica.\\
\end{itemize}

\item ¿Qué mide la sonda de presión estática? [1 min]
\begin{itemize}
    \item[A)] La presión total del aire.
    \item[B)] La presión dinámica del aire.
    \item[C)] La presión atmosférica circundante.\\
\end{itemize}

\item ¿Cuál es el propósito de la sonda Kiel? [1 min] 
    \begin{itemize}
    \item[A)] Medir la temperatura del aire. 
    \item[B)] Reducir los efectos del incremento de ángulo de ataque en la medición de presión. 
    \item[C)] Medir la presión dinámica. \\
    \end{itemize}

\item  ¿Qué indica el indicador de velocidad vertical (VSI)? [1 min]
    \begin{itemize}
    \item[A)] La velocidad horizontal del avión.
    \item[B)] La tasa de ascenso o descenso de la aeronave.
    \item[C)] La velocidad del viento.\\
    \end{itemize}

\item ¿Cuál es el principal componente de un altímetro barométrico? [1 min]
    \begin{itemize}
    \item[A)] Tubo de Pitot.
    \item[B)] Sensor GPS.
    \item[C)] Cápsulas aneroides.\\
    \end{itemize}

\item ¿Qué representan las marcas de color (verde, blanco, amarillo, rojo) en el indicador de velocidad aérea (ASI)? [1 min]
     \begin{itemize}
    \item[A)] Zonas climáticas.
    \item[B)] Limitaciones de velocidad crítica del avión.
    \item[C)] Nivel de combustible.\\
    \end{itemize}
    
\item En un variómetro de tipo pasivo, el objetivo de la tasa del flujo (Q) en la entrada de la carcasa es: [1 min]
     \begin{itemize}
	\item[A)] Evitar los efectos de la temperatura de remanso en la indicación.
	\item[B)] Reducir los efectos debidos a la altura en la que se encuentre el avión.
	\item[C)] La A) y la B) son correctas.\\
\end{itemize}

\item ¿Qué tipo de datos se miden por el ADC (Central Air Data Computer)? [1 min] 
    \begin{itemize}
    \item[A)] Presiones,temperaturas y densidades del aire. 
    \item[B)] Presiones del agua y viento.
    \item[C)] Temperaturas internas del motor.\\
    \end{itemize}

\item  ¿Qué indica el indicador de velocidad vertical (VSI)? [1 min]
    \begin{itemize}
    \item[A)] La velocidad horizontal del avión.
    \item[B)] La tasa de ascenso o descenso de la aeronave.
    \item[C)] La velocidad del viento.\\
    \end{itemize}

\item ¿Qué indica el número de Mach? [1 min]
    \begin{itemize}
    \item[A)] La altitud del avión. 
    \item[B)] La temperatura del aire.
    \item[C)] La relación entre la velocidad del avión y la velocidad del sonido.\\
    \end{itemize}

\item  ¿Cuál es la ventaja de utilizar la velocidad indicada del aire (IAS)? [1 min]
    \begin{itemize}
    \item[A)] Es fácil de convertir en TAS.
    \item[B)] Se obtiene a partir de datos del GPS.
    \item[C)] Todos los parametros de vuelo importantes se basan en la IAS.\\
    \end{itemize}
\end{enumerate}



\begin{nosangria}
\Large{\textbf{Elige entre VERDADERO o FALSO de las siguientes afirmaciones} [15 min]}
\end{nosangria}

\begin{enumerate}
\item (V/F) No se puede conseguir la información sobre la presión atmosférica, temperatura y densisdad del aire sin la ayuda de la ISA. 

\item (V/F) La atmósfera estándar internacional (ISA) asume la presencia de polvo y vapor de agua.

\item (V/F) La sonda de presión estática se coloca en la punta del avión para obtener mediciones precisas.

\item (V/F) La presión dinámica es la presión del aire en reposo.

\item (V/F) El VSI (indicador de velocidad vertical) mide la tasa de cambio de la presión estática.

\item (V/F) El número de Mach indica la altitud del avión.

\item (V/F) En un avión, el ASI (indicador de velocidad aérea) siempre indica la velocidad verdadera (TAS).

\item (V/F) El ADC reemplaza completamente a los sensores de presión estática y total, ya que genera estas presiones de forma digital.

\item (V/F) La sustitución de un ADC requiere procedimientos complejos que incrementan el tiempo de mantenimiento.

\item (V/F) La relación entre la velocidad del avión y la velocidad del sonido depende de la altura.

\item (V/F) Las lecturas del barómetro son lineales al cambio de altura.\\
\end{enumerate}


\begin{nosangria}
\Large{\textbf{PROBLEMAS}}
\end{nosangria}

\textbf{1.1.}  
\begin{enumerate}[label=(\alph*)]
    \item A una altitud de 20,000 metros (65,617 pies), la medición de presión estática es de 41.41 mm de Hg. ¿Cuál es la temperatura estática?
    \item Si el número de Mach es 4.0 a esta altitud, ¿cuál es la presión de impacto \( q_c \) en mm de Hg?
    \item Bajo las condiciones anteriores, ¿cuál es la velocidad verdadera en nudos?
\end{enumerate}

\textit{Resp.:} (a) 235 K; (b) 831 mm Hg; (c) 2388 nudos.

\vspace{1em}

\textbf{1.2.}  
\begin{enumerate}[label=(\alph*)]
    \item Si una aeronave vuela a Mach = 0.8, ¿cuál es la razón de presión de impacto a presión estática \( q_c / p_s \)?
    \item Si la aeronave tiene un error de fuente estática sin corregir de ±5\% y un error del tubo pitot de ±5\%, ¿cuál es el rango de números de Mach calculados?
\end{enumerate}

\textit{Resp.:} (a) \( q_c / p_s = 0.524 \); (b) \( M = 0.766 \) a \( 0.835 \).
