
\chapter*{Cuestionario} % Capítulo sin número
\addcontentsline{toc}{chapter}{Cuestionario Bloque 2} % Añadirlo manualmente al índice
\setchapterimage{orange8.jpg} % Chapter heading image
\chapterspaceabove{6.75cm} % Whitespace from the top of the page to the chapter title on chapter pages
\chapterspacebelow{7.25cm} % Amount of vertical whitespace from the top margin to the start of the text on chapter pages
\pagestyle{empty}
\begin{nosangria}
\Large{\textbf{Elige UNA respuesta para cada una de las preguntas} [18 min]}
\end{nosangria}

\begin{enumerate}
\item ¿Cuáles son las dos propiedades fundamentales en las que se basan los giróscopos mecánicos? [1 min]

	\begin{itemize}
		\item[A)] Precesión y fricción.
		\item[B)] Momento angular y vacío.
		\item[C)] Rigidez y precesión.\\
	\end{itemize}

\item La propiedad de un giróscopo de mantener su eje de giro fijo en el espacio se denomina: [1 min]

	\begin{itemize}
		\item[A)] Precesión.
		\item[B)] Momento de inercia.
		\item[C)] Rigidez.\\
	\end{itemize}

\item La rigidez de un giróscopo aumenta si: [1 min]

	\begin{itemize}
		\item[A)] Disminuye su velocidad angular (\(\omega\)).
		\item[B)] Aumenta su momento angular (\(L = I \omega\)).
		\item[C)] Se aplica un par perpendicular.\\
	\end{itemize}

\item ¿Qué es la precesión en un giróscopo? [1 min]

	\begin{itemize}
		\item[A)] La tendencia a mantener su orientación.
		\item[B)] El movimiento del eje de giro en respuesta a un par aplicado perpendicularmente.
		\item[C)] El desgaste de los rodamientos.\\
	\end{itemize}

\item La "Deriva Verdadera" en un giróscopo mecánico se debe principalmente a:[1 min]

	\begin{itemize}
		\item[A)]  La rotación de la Tierra y el movimiento del avión.
		\item[B)] La fricción en los rodamientos y el desequilibrio del rotor.
		\item[C)]  La aplicación de un par externo intencionado.\\
	\end{itemize}

\item La "Deriva Astronómica" es un tipo de deriva aparente causada por: [1 min]

	\begin{itemize}
		\item[A)]  El movimiento del avión sobre la superficie terrestre.
		\item[B)] La fricción interna del instrumento.
		\item[C)]  La rotación de la Tierra.\\
	\end{itemize}

\item ¿Cuál es una fuente de energía común para los giróscopos en aviación general, según el texto? [1 min]

	\begin{itemize}
		\item[A)]  Baterías de litio exclusivamente.
		\item[B)] Un sistema de vacío accionado por el motor.
		\item[C)]  Energía nuclear a pequeña escala.\\
	\end{itemize}

\item ¿Qué instrumento giroscópico utiliza principalmente la propiedad de rigidez para mantener una referencia de rumbo horizontal? [1 min]

	\begin{itemize}
		\item[A)] Indicador de Actitud (Horizonte Artificial).
		\item[B)] Indicador de Giro.
		\item[C)] Indicador de Rumbo (Heading Indicator).\\
	\end{itemize}

\item El Indicador de Actitud (Horizonte Artificial) muestra principalmente: [1 min]

	\begin{itemize}
		\item[A)] La velocidad de giro y el deslizamiento.
		\item[B)] El rumbo magnético y la deriva.
		\item[C)] La inclinación lateral (alabeo) y el cabeceo de la aeronave.\\
	\end{itemize}

\item ¿Cuál es la diferencia principal entre un Indicador de Giro y un Coordinador de Giro? [1 min]

	\begin{itemize}
		\item[A)] El Indicador de Giro usa electricidad y el Coordinador usa vacío.
		\item[B)] El Coordinador de Giro también detecta la tasa de alabeo (roll rate), mientras que el Indicador de Giro solo mide la guiñada (yaw rate).
		\item[C)] El Coordinador de Giro se basa en la rigidez, el Indicador de Giro en la precesión.\\
	\end{itemize}

\item ¿Cuál es el principio físico fundamental en el que se basan los giróscopos ópticos como el FOG y el RLG? [1 min]

	\begin{itemize}
		\item[A)] Efecto Coriolis.
		\item[B)] Efecto Sagnac.
		\item[C)] Efecto Hall.\\
	\end{itemize}

\item Según el efecto Sagnac, cuando un lazo cerrado se encuentra rotando, ¿qué ocurre con los dos haces de luz que viajan en sentidos opuestos? [1 min]

	\begin{itemize}
		\item[A)] Ambos aumentan su velocidad.
		\item[B)] Uno recorre una distancia ligeramente mayor y el otro una menor, llegando a destiempo.
		\item[C)] Se anulan mutuamente por interferencia destructiva total.\\
	\end{itemize}

\item En un Giróscopo de Fibra Óptica (FOG), ¿qué parámetro se mide para determinar la velocidad de rotación? [1 min]

	\begin{itemize}
		\item[A)] La intensidad de la luz.
		\item[B)] El cambio de frecuencia de la luz.
		\item[C)] El desplazamiento de fase (\(\Delta \phi\)) entre los haces de luz.\\
	\end{itemize}

\item ¿Qué problema común afecta a los Giróscopos de Anillo Láser (RLG) a bajas velocidades de rotación? [1 min]

	\begin{itemize}
		\item[A)] Sobrecalentamiento del gas inerte.	
		\item[B)] Bloqueo de fase ("lock-in").
		\item[C)] Interferencia con el campo magnético terrestre.\\
	\end{itemize}

\item ¿Qué técnica se utiliza en los RLG para evitar el problema de "lock-in"? [1 min]

	\begin{itemize}
		\item[A)] Uso de fibra óptica en lugar de cavidad.	
		\item[B)] Dithering (oscilación controlada de la cavidad).
		\item[C)] Aislamiento magnético.\\
	\end{itemize}

\item ¿En qué principio físico se basa el funcionamiento de los giróscopos MEMS? [1 min]

	\begin{itemize}
		\item[A)] Efecto Sagnac.
		\item[B)] Efecto Coriolis.
		\item[C)] Inducción electromagnética.\\
	\end{itemize}

\item ¿Qué magnitud física se mide directamente en los giróscopos y acelerómetros MEMS para detectar el movimiento o la rotación? [1 min]

	\begin{itemize}
		\item[A)] La resistencia eléctrica.
		\item[B)] El campo magnético.
		\item[C)] La capacitancia.\\
	\end{itemize}

\item ¿Qué es un sistema AHRS mencionado en el contexto de los MEMS? [1 min]
	\begin{itemize}
		\item[A)]  Un sistema de Alerta de Humo y Radiación.
		\item[B)] Un Sistema de Referencia de Actitud y Rumbo (Attitude and Heading Reference System).
		\item[C)] Un Algoritmo de Sincronización Remota.\\
	\end{itemize}
\end{enumerate}





\begin{nosangria}
\Large{\textbf{Elige entre VERDADERO o FALSO de las siguientes afirmaciones} [15 min]}
\end{nosangria}

\begin{enumerate}
\item (V/F) La precesión ocurre en la misma dirección en que se aplica el par.

\item (V/F) Un giróscopo libre (Free gyro) tiene utilidad práctica directa en los instrumentos de vuelo de un avión.

\item (V/F) Es necesario corregir periódicamente el Indicador de Rumbo (HI) usando la brújula magnética debido a la deriva.

\item (V/F) Los giróscopos atados (Tied gyros) se usan en indicadores de dirección y actitud.

\item (V/F) El Indicador de Actitud (AI) funciona correctamente sin importar cuánto se incline o cabecee el avión.

\item (V/F) La bola en el indicador de deslizamiento (inclinómetro) del Coordinador de Giro ayuda a detectar si el giro es coordinado

\item (V/F) Una tasa de giro estándar (Giro 1) corresponde a 1 grado por segundo.

\item (V/F) El principal problema de los giróscopos mecánicos es la deriva, que puede ser significativa en sistemas INS.

\item (V/F) En el efecto Sagnac, la diferencia de tiempo (\(\Delta t\)) es inversamente proporcional a la velocidad angular (\(\omega\)).

\item (V/F) Los giróscopos de fibra óptica (FOG) utilizan una cavidad llena de gas Helio-Neón para guiar la luz.

\item (V/F) Los sistemas MEMS destacan por su gran tamaño y alto consumo de energía en comparación con los giróscopos ópticos.

\item (V/F) Un solo sensor acelerómetro MEMS es suficiente para medir la aceleración absoluta en las tres dimensiones (X, Y, Z).\\
\end{enumerate}


	
	
	
	
	